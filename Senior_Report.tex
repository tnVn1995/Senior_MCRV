\documentclass[]{article}
\usepackage{lmodern}
\usepackage{amssymb,amsmath}
\usepackage{ifxetex,ifluatex}
\usepackage{fixltx2e} % provides \textsubscript
\ifnum 0\ifxetex 1\fi\ifluatex 1\fi=0 % if pdftex
  \usepackage[T1]{fontenc}
  \usepackage[utf8]{inputenc}
\else % if luatex or xelatex
  \ifxetex
    \usepackage{mathspec}
  \else
    \usepackage{fontspec}
  \fi
  \defaultfontfeatures{Ligatures=TeX,Scale=MatchLowercase}
\fi
% use upquote if available, for straight quotes in verbatim environments
\IfFileExists{upquote.sty}{\usepackage{upquote}}{}
% use microtype if available
\IfFileExists{microtype.sty}{%
\usepackage{microtype}
\UseMicrotypeSet[protrusion]{basicmath} % disable protrusion for tt fonts
}{}
\usepackage[margin=1in]{geometry}
\usepackage{hyperref}
\hypersetup{unicode=true,
            pdftitle={Senior Report},
            pdfborder={0 0 0},
            breaklinks=true}
\urlstyle{same}  % don't use monospace font for urls
\usepackage{graphicx,grffile}
\makeatletter
\def\maxwidth{\ifdim\Gin@nat@width>\linewidth\linewidth\else\Gin@nat@width\fi}
\def\maxheight{\ifdim\Gin@nat@height>\textheight\textheight\else\Gin@nat@height\fi}
\makeatother
% Scale images if necessary, so that they will not overflow the page
% margins by default, and it is still possible to overwrite the defaults
% using explicit options in \includegraphics[width, height, ...]{}
\setkeys{Gin}{width=\maxwidth,height=\maxheight,keepaspectratio}
\IfFileExists{parskip.sty}{%
\usepackage{parskip}
}{% else
\setlength{\parindent}{0pt}
\setlength{\parskip}{6pt plus 2pt minus 1pt}
}
\setlength{\emergencystretch}{3em}  % prevent overfull lines
\providecommand{\tightlist}{%
  \setlength{\itemsep}{0pt}\setlength{\parskip}{0pt}}
\setcounter{secnumdepth}{0}
% Redefines (sub)paragraphs to behave more like sections
\ifx\paragraph\undefined\else
\let\oldparagraph\paragraph
\renewcommand{\paragraph}[1]{\oldparagraph{#1}\mbox{}}
\fi
\ifx\subparagraph\undefined\else
\let\oldsubparagraph\subparagraph
\renewcommand{\subparagraph}[1]{\oldsubparagraph{#1}\mbox{}}
\fi

%%% Use protect on footnotes to avoid problems with footnotes in titles
\let\rmarkdownfootnote\footnote%
\def\footnote{\protect\rmarkdownfootnote}

%%% Change title format to be more compact
\usepackage{titling}

% Create subtitle command for use in maketitle
\providecommand{\subtitle}[1]{
  \posttitle{
    \begin{center}\large#1\end{center}
    }
}

\setlength{\droptitle}{-2em}

  \title{Senior Report}
    \pretitle{\vspace{\droptitle}\centering\huge}
  \posttitle{\par}
    \author{}
    \preauthor{}\postauthor{}
    \date{}
    \predate{}\postdate{}
  

\begin{document}
\maketitle

\hypertarget{testing-for-marginal-independence-between-two-categorical-variables-with-multiple-responses}{%
\section{Testing for Marginal Independence between Two Categorical
Variables with Multiple
Responses}\label{testing-for-marginal-independence-between-two-categorical-variables-with-multiple-responses}}

Mind and body practices may be used to improve health and well-being or
to help manage symptoms of health problems. The 2012 National Health
Interview Survey collected information about Americans' top 3 commonly
used modalities and whether using each modality is because of the
recommendation from any doctor, family member, or friend. There are
statistical methods to study the association between two multiple
response categorical variables, i.e.~modality selection and
recommendation in the survey. However, due to the way the questions were
designed in the survey, no information was collected about failure
recommendation. No existing efficient statistical methods can be
directly applied to the scenario. In this study, we proposed a modified
Pearson chi-square statistic to analyze the special data structure in
the survey. Simulations were conducted to evaluate the proposed method.

\hypertarget{data-structure}{%
\subsection{Data Structure}\label{data-structure}}

\begin{verbatim}
# A tibble: 6 x 15
  TP1_REC1 TP1_REC2 TP1_REC3 TP1_REC4 TP2_REC1 TP2_REC2 TP2_REC3 TP2_REC4
  <dbl+lb> <dbl+lb> <dbl+lb> <dbl+lb> <dbl+lb> <dbl+lb> <dbl+lb> <dbl+lb>
1 NA       NA        NA       NA      NA        NA       NA       NA     
2  2 [No]   2 [No]    2 [No]   2 [No]  2 [No]    2 [No]   2 [No]   2 [No]
3  1 [Yes]  2 [No]    2 [No]   2 [No]  1 [Yes]   2 [No]   2 [No]   2 [No]
4 NA       NA        NA       NA      NA        NA       NA       NA     
5  2 [No]   1 [Yes]   2 [No]   2 [No]  2 [No]    2 [No]   2 [No]   2 [No]
6 NA       NA        NA       NA      NA        NA       NA       NA     
# ... with 7 more variables: TP3_REC1 <dbl+lbl>, TP3_REC2 <dbl+lbl>,
#   TP3_REC3 <dbl+lbl>, TP3_REC4 <dbl+lbl>, ALT_TP31 <dbl+lbl>,
#   ALT_TP32 <dbl+lbl>, ALT_TP33 <dbl+lbl>
\end{verbatim}

Above are the first five observations of the data. The data contains 16
columns. Column 13-15 show the top 3 choices of alternative therapies
for individuals (coded from 1-16,corresponding to 16 alternative
therapies). The first 12 columns show the recommendation sources for
each individual. Each top choice for an individual can be influenced by
one or more of the recommendation sources (e.g family, friends). The
first four columns correspond to 4 recommendation sources for the 1st
choice of alternative therapy. Similarly, the next 4 columns correspond
to the 2nd choice alternative therapy. The last 4 columns are the
recommendation sources for the 3rd top choice. The research question is
whether top 3 alternative therapies are independent with recommendation
sources or not. Our research hypothesis is then:

\(H_{0}:\) Top 3 Alternative Therapies and Recommendation Sources are
independent \(H_{1}:\) They are not independent

\begin{verbatim}
Warning: `as.tibble()` is deprecated, use `as_tibble()` (but mind the new semantics).
This warning is displayed once per session.
\end{verbatim}

\begin{verbatim}
# A tibble: 7,851 x 12
   TP1_REC1 TP1_REC2 TP1_REC3 TP1_REC4 TP2_REC1 TP2_REC2 TP2_REC3 TP2_REC4
   <dbl+lb> <dbl+lb> <dbl+lb> <dbl+lb> <dbl+lb> <dbl+lb> <dbl+lb> <dbl+lb>
 1   2 [No]  2 [No]   2 [No]    2 [No]   2 [No]  2 [No]   2 [No]    2 [No]
 2   2 [No]  1 [Yes]  2 [No]    2 [No]   2 [No]  2 [No]   2 [No]    2 [No]
 3   2 [No]  1 [Yes]  2 [No]    2 [No]   2 [No]  1 [Yes]  2 [No]    2 [No]
 4   2 [No]  2 [No]   2 [No]    2 [No]   2 [No]  2 [No]   2 [No]    2 [No]
 5   2 [No]  2 [No]   2 [No]    2 [No]  NA      NA       NA        NA     
 6   2 [No]  1 [Yes]  2 [No]    2 [No]   2 [No]  2 [No]   1 [Yes]   2 [No]
 7   2 [No]  2 [No]   2 [No]    2 [No]   2 [No]  2 [No]   2 [No]    2 [No]
 8   2 [No]  1 [Yes]  2 [No]    2 [No]   2 [No]  2 [No]   1 [Yes]   2 [No]
 9   2 [No]  2 [No]   2 [No]    2 [No]   2 [No]  2 [No]   2 [No]    2 [No]
10   2 [No]  2 [No]   1 [Yes]   2 [No]   2 [No]  2 [No]   1 [Yes]   2 [No]
# ... with 7,841 more rows, and 4 more variables: TP3_REC1 <dbl+lbl>,
#   TP3_REC2 <dbl+lbl>, TP3_REC3 <dbl+lbl>, TP3_REC4 <dbl+lbl>
\end{verbatim}

\hypertarget{test}{%
\section{Test}\label{test}}


\end{document}
